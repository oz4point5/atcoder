\documentclass[a4paper,10.5pt]{jarticle}

\usepackage[truedimen,top=25truemm,bottom=30truemm,hmargin=25truemm]{geometry}
\usepackage{calc}
\usepackage[dvipdfmx]{graphicx}
\usepackage{pxrubrica}
\usepackage[dvipdfmx]{hyperref}
\usepackage{pxjahyper}

\begin{document}

%
%	表紙を必要としないもの用のTeX雛形(作成:@oz4point5)
%	おおむね社会学会の論文用の書式に従っている
%
%	文字組他に関しては以下のウェブページを参考にした
%	https://texwiki.texjp.org/?geometry
%

\makeatletter
\newcount\@chars\newcount\@lines
\@chars=40                      % 1行の文字数
\@lines=40                      % 1ページの行数
\newdimen\@kanjiskip
\@kanjiskip=\dimexpr(\textwidth-1zw*\@chars)/\numexpr\@chars-1
\newdimen\@@kanjiskip
\@@kanjiskip=\dimexpr\@kanjiskip/10
\setlength{\@tempdima}{1pt*\ratio{\dimexpr\textheight/\@lines}{\baselineskip}}
\renewcommand{\baselinestretch}{\strip@pt\@tempdima}\selectfont
\kanjiskip=\@kanjiskip plus \@@kanjiskip minus \@@kanjiskip
\parindent=\dimexpr 1zw+2truept
\parindent=\dimexpr\parindent+\@kanjiskip
\makeatother

%
%	タイトルなど
%

\title{Pythonチュートリアル}
\author{@oz4point5}
\date{}
\maketitle

このpdfは@oz4point5が自習用にhttps://docs.python.org/ja/3/tutorial/をまとめたものです。著作権は元著者に帰属します。\\
\\
Pythonは、
\begin{itemize}
\item 効率的な高レベルデータ構造
\item シンプルで効果的なオブジェクト指向プログラミング機構
\item 洗練された文法
\item 動的なデータ型付け
\item インタープリタ
\item スクリプティグや高速アプリケーション開発(Rapid Application Development: RAD)に理想的
\end{itemize}

\tableofcontents

%
%	以下本文
%

\newpage
\section{やる気を高めよう}
Pythonは、
\begin{itemize}
\item 簡単に利用できるが、本物のプログラミング言語であり、シェルスプリクトやバッチファイルよりも多くの機構があり、大きなプログラムの開発にも適しています。
\item プログラムをモジュールに分割して、他のPythonプログラムで再利用できます。
\item インタプリタ言語です。コンパイルやリンクの必要がないので、プログラムを開発する際にかなりの時間を節約できます。
\item 名前は"モンティパイソンの空飛ぶサーカス"から取られている。
\end{itemize}

\section{Pythonインタプリタを使う}
\subsection{インタプリタを起動する}
\begin{description}
\item[インタプリタを起動する]~\\
\begin{verbatim}
  python3.7
\end{verbatim}
\item[インタプリタを終了する]~\\
ファイル終端文字(Ctrl-D, Ctrl-Z)を1次プロンプトへ入力または、
\begin{verbatim}
  quit()
\end{verbatim}
\end{description}

\end{document}
